%\usepackage[cm-default]{fontspec}
\usepackage{fontspec}
\emfontdeclare{\itshape,\upshape,\itshape}
\setmainfont{CMU Serif}
\setsansfont{CMU Sans Serif}
\setmonofont{CMU Typewriter Text}
\usepackage{xunicode}
\usepackage{xeCJK}\setCJKmainfont[BoldFont={MATHIST_BOLD_FONT},ItalicFont={MATHIST_ITALIC_FONT}]{MATHIST_MAIN_FONT}
\setCJKsansfont{MATHIST_SANS_FONT}
\setCJKmonofont{MATHIST_MONO_FONT} % 打印机字体, 一般用于 verbatim 之类的环境, 或者通过 \ttfamily 调用
\newCJKfontfamily\fontfortheorem[BoldFont={MATHIST_THEOREM_BOLD_FONT},ItalicFont={MATHIST_THEOREM_ITALIC_FONT}]{MATHIST_THEOREM_MAIN_FONT} % 新定义用于定理环境的字体

\usepackage{enumerate}

\usepackage{amsthm}\newtheoremstyle{Theorem}{3pt}{3pt}{\fontfortheorem}{}{\bfseries}{}{1em}{}
%\theoremstyle{Theorem} \newtheorem{Theorem}{定理}[section]%按节编号
\theoremstyle{Theorem} \newtheorem{Theorem}{定理}[chapter]%按章编号
\newtheoremstyle{Definition}{3pt}{3pt}{\fontfortheorem}{}{\bfseries}{}{1em}{}
\theoremstyle{Definition} \newtheorem{Definition}[Theorem]{定义}
\newtheoremstyle{Example}{3pt}{3pt}{\fontfortheorem}{}{\bfseries}{}{1em}{}
\theoremstyle{Example} \newtheorem{Example}[Theorem]{例}
\theoremstyle{Example} \newtheorem{Note}[Theorem]{注记}
\theoremstyle{Theorem} \newtheorem{Lemma}[Theorem]{引理}
\theoremstyle{Theorem} \newtheorem{Corollary}[Theorem]{推论}
\theoremstyle{Theorem} \newtheorem{Proposition}[Theorem]{命题}
\theoremstyle{Theorem} \newtheorem{Conjecture}[Theorem]{猜想}
\theoremstyle{Theorem} \newtheorem{Fact}[Theorem]{事实}
\theoremstyle{Example} \newtheorem{Problem}[Theorem]{问题}
\theoremstyle{Example} \newtheorem{Exercise}[Theorem]{习题}
\theoremstyle{Example} \newtheorem{Claim}[Theorem]{断言}

\theoremstyle{Theorem} \newtheorem*{Theorem*}{定理}
\theoremstyle{Definition} \newtheorem*{Definition*}{定义}
\theoremstyle{Example} \newtheorem*{Example*}{例}
\theoremstyle{Example} \newtheorem*{Note*}{注记}
\theoremstyle{Theorem} \newtheorem*{Lemma*}{引理}
\theoremstyle{Theorem} \newtheorem*{Corollary*}{推论}
\theoremstyle{Theorem} \newtheorem*{Proposition*}{命题}
\theoremstyle{Theorem} \newtheorem*{Conjecture*}{猜想}
\theoremstyle{Theorem} \newtheorem*{Fact*}{事实}
\theoremstyle{Example} \newtheorem*{Problem*}{问题}
\theoremstyle{Example} \newtheorem*{Exercise*}{习题}
\theoremstyle{Example} \newtheorem*{Claim*}{断言}

% 致谢环境
\theoremstyle{Example} \newtheorem*{Acknowledgements}{致谢}

% 证明与提示
\newenvironment{Proof}[1][证明]
{\noindent\normalfont {\bfseries #1}\hphantom{空}}{\hfill\ensuremath{\square}}

\newenvironment{HINT}[1][提示]{\noindent\normalfont {\bfseries #1}\hphantom{空}}{\hfill\ensuremath{\square}}



%格式化引用

\usepackage{prettyref}

\newrefformat{Theorem}{定理~\ref{#1}~}
\newrefformat{Definition}{定义~\ref{#1}~}
\newrefformat{Example}{例~\ref{#1}~}
\newrefformat{Remark}{注记~\ref{#1}~}
\newrefformat{Lemma}{引理~\ref{#1}~}
\newrefformat{Corollary}{推论~\ref{#1}~}
\newrefformat{Proposition}{命题~\ref{#1}~}
\newrefformat{Conjecture}{猜想~\ref{#1}~}
\newrefformat{Fact}{事实~\ref{#1}~}
\newrefformat{Problem}{问题~\ref{#1}~}
\newrefformat{Exercise}{习题~\ref{#1}~}
\newrefformat{Claim}{断言~\ref{#1}~}
\newrefformat{Picture}{图~\ref{#1}~}
\newrefformat{fig}{图~\ref{#1}~}
\newrefformat{tab}{表~\ref{#1}~}
\newrefformat{fn}{第~\pageref{#1}~页的脚注~\ref{#1}~}
\newrefformat{part}{第~\ref{#1}~部分}
\newrefformat{chap}{第~\ref{#1}~章}
\newrefformat{sec}{~\ref{#1}~节}
\newrefformat{sub}{~\ref{#1}~小节}
\newrefformat{subsec}{~\ref{#1}~小节}


\usepackage{caption}
\renewcommand{\fnum@figure}{图~\thefigure}

\renewcommand{\tablename}{表}

%\renewcommand{\abstractname}{摘要}
\renewcommand{\contentsname}{目\quad 录}
\renewcommand{\indexname}{索\quad 引}

\ifx\nomname\undefined
\else
\renewcommand{\nomname}{符号表}
\fi

\ifx\pagedeclaration\undefined
\else
\renewcommand{\pagedeclaration}[1]{, 在第~#1~页}
\fi


%收集摘要
\newenvironment{Abstract_mathist}[5]{\noindent\normalfont {\bfseries 标题:} #1\\
	\noindent {\bfseries 作者:} #2\\
	\noindent {\bfseries 日期:} #3 | {\bfseries 编号:} #4\\
	\noindent {\bfseries 关键词:} #5
	
	\noindent {\bfseries 摘要:} }{
		
		}

% 允许行间公式跨页, 参数为 1,2,3,4 的一个, 越大则越可能跨页
\allowdisplaybreaks[1]

% 显示用于交叉引用的标签名, 这里的宏包应该最后加载
% \usepackage[notref,notcite]{showkeys} % 旧的解法
% 现在用 showlabels 宏包, 注意它不兼容 multicol, wrapfig 等宏包.
\ifx\DoNotUseShowlabels\undefined
\usepackage[left]{showlabels}
\else
\fi

% 修改页面背景色
% 默认的白底黑字适合明亮的环境, 当然很明亮也不行, 毕竟主要用的是衬线字体. 下面这个设置适合较暗的环境, 例如阴天.
%\usepackage{color}
%\definecolor{page_backgroundcolor}{rgb}{0.992188, 0.964844, 0.890625}
%\pagecolor{page_backgroundcolor}

% 定义数学宏
\input{MATHIST_DOCUMENT_PATH/Other/MathMacros/MathMacros.tex}

% 全局设定浮动项的位置
\usepackage{newfloat}
\SetupFloatingEnvironment{figure}{placement=htb}
\SetupFloatingEnvironment{table}{placement=htb}

% 手动载入某些宏包
\usepackage{mathtools}
\usepackage{tikz}
\usepackage{multicol}

% 让 url 断行
\def\UrlAlphabet{%
	\do\a\do\b\do\c\do\d\do\e\do\f\do\g\do\h\do\i\do\j%
	\do\k\do\l\do\m\do\n\do\o\do\p\do\q\do\r\do\s\do\t%
	\do\u\do\v\do\w\do\x\do\y\do\z\do\A\do\B\do\C\do\D%
	\do\E\do\F\do\G\do\H\do\I\do\J\do\K\do\L\do\M\do\N%
	\do\O\do\P\do\Q\do\R\do\S\do\T\do\U\do\V\do\W\do\X%
	\do\Y\do\Z}
\def\UrlDigits{\do\1\do\2\do\3\do\4\do\5\do\6\do\7\do\8\do\9\do\0}
\g@addto@macro{\UrlBreaks}{\UrlOrds}
\g@addto@macro{\UrlBreaks}{\UrlAlphabet}
\g@addto@macro{\UrlBreaks}{\UrlDigits}

% preamble 最后才插入的内容
\AtBeginDocument{
	\renewcommand{\bibtitlefont}{\itshape }%
	\def\UrlFont{\ttfamily}%
	\usepackage{xpatch}%
	\xapptobibmacro{finentry}{\addspace \printfield{license} \addperiod}{}{} % 配合 add-license.dbx 与 biblatex 宏包选项 datamodel=add-license, 在参考文献列表中注明版权信息
}